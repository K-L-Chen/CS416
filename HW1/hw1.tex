\documentclass[12pt]{article} % 

\usepackage[utf8]{inputenc}
\usepackage{graphicx}
\usepackage{verbatim}
\usepackage{amssymb}
\usepackage{amsthm}
\usepackage{amsfonts}
\usepackage{amsmath}
\usepackage{tikz}

\setlength{\oddsidemargin}{-0.15in}
\setlength{\topmargin}{-0.5in}
\setlength{\textwidth}{6.5in}
\setlength{\textheight}{9in}


\newcommand{\Z}{\mathbb{Z}}
\newcommand{\R}{\mathbb{R}}
\newcommand{\D}{\partial}
\begin{document} 
\noindent
\textbf{CSCI 416 - Intro to Machine Learning \quad 
Homework 9/30 \hfill Kyle Chen}\\
\begin{center}
  (Due September 30)
\end{center}
\bigskip

\begin{flushleft} %The ``flushleft'' environment makes sure everything is left aligned.

\textbf{Question 1} 
\begin{enumerate}
	\item[(1)]
		\begin{align*}
			\frac{\D}{\D u}f(u,v) &= 16uv^4+6
		\end{align*}
	\item[(2)]
		\begin{align*}
			\frac{\D}{\D v}f(u,v) &= 32u^2v^3+12v^2
		\end{align*}
	\item[(3)]
		\begin{align*}
			\frac{\D}{\D u}g(u,v,w) &= \frac{x}{u}+yvw^3
		\end{align*}
	\item[(4)]	
		\begin{align*}
			\frac{\D}{\D v}g(u,v,w) &= yuw^3
		\end{align*}
	\item[(5)]
		\begin{align*}
			\frac{\D}{\D w}g(u,v,w) &= 3yuvw^2
		\end{align*}
	\item[(6)]
		\begin{align*}
			\frac{\D}{\D u}h(u,v) &= \sum^m_{i = 1}{(x^{(i)})^2 u+x^{(i)}y^{(i)}v}
		\end{align*}
	\item[(7)]
		\begin{align*}
			\frac{\D}{\D v}h(u,v) &= \sum^m_{i = 1}{x^{(i)}y^{(i)}u+(y^{(i)})^2v}
		\end{align*}
\end{enumerate}

%\vspace{1cm}
\newpage
\textbf{Question 2} 
\begin{enumerate}
	\item[(1)] Negative.
	\item[(2)] Negative.
	\item[(3)] Positive.
	\item[(4)]	Negative.
	\item[(5)] $u = 1, v = 2$ 
\end{enumerate}

\vspace{1cm}

\textbf{Question 3} 
\begin{enumerate}
	\item[(1)]
		\[
		\left[
			{\begin{array}{cc}
		   		3 & -1 \\
				2 & 5 \\
				-2 & 2
		  \end{array} } 
		\right]
		\cdot
		\left[
			{\begin{array}{cc}
		   		u & a \\
				v & b
		  \end{array} } 
		\right]
		=
		\left[
			{\begin{array}{cc}
		   		3u - v & 3a - b \\
				2u + 5v & 2a+5b \\
				-2u + 2v & -2a + 2b
		  \end{array} } 
		\right]
		\]
	\item[(2)] Yes, product $AB \in \R^{2\times4}$
	\item[(3)] No.
	\item[(4)]	 $y^TA$ is a row vector, $y^TA \in \R^{1\times2}$
	\item[(5)] $Ax$ is a column vector, $Ax \in \R^3$
	\item[(6)] Since we know that the zero vector $0$ is a row vector, we can proceed as follows:
		\begin{align*}
			(Bx + y)^TA^T &= 0 &&\\
			(Bx + y)^TA^T\cdot (A^T)^-1 &= 0\cdot(A^T)^-1 &&\\
			(Bx + y)^T(A^T\cdot (A^T)^-1) &= 0\cdot(A^T)^-1 &&\\
			(Bx + y)^T(I^{n\times n}) &= 0 &&\\
			(Bx + y)^T &= 0 &&\\
			((Bx + y)^T)^T &= 0^T &&\\
			Bx + y &= 0^T &&\\
			Bx &= 0^T - y &&\\
			Bx &= -y &&\\
			B^{-1}\cdot Bx &= B^{-1}\cdot(-y) &&\\
			(B^{-1}\cdot B)x &= -B^{-1}y &&\\
			I^{n\times n}x &= -B^{-1}y &&\\
			x &= -B^{-1}y
		\end{align*}
	and we are done.
\end{enumerate}

\vspace{1cm}

\textbf{Question 4}
\begin{verbatim}
Part 1:
 A = 
 [[-2 -3]
 [ 1  0]] 
B = 
 [[-1  1]
 [ 1  0]] 
x = 
 [[-1]
 [ 1]]
Part 2:
 C = 
 [[-0.          1.        ]
 [-0.33333333 -0.66666667]]
Part 3:
 AC = 
 [[ 1.00000000e+00 -1.11022302e-16]
 [ 0.00000000e+00  1.00000000e+00]] 
CA = 
 [[1. 0.]
 [0. 1.]]
Part 4: 
 Ax = 
 [[-1]
 [-1]]
Part 5:
 A^(T) A = 
 [[5 6]
 [6 9]]
Part 6:
 Ax - Bx = 
 [[-3]
 [ 0]]
Part 7:
 ||x|| = 
 [[1.41421356]]
Part 8:
 ||Ax - Bx|| = 
 3.0
Part 9:
 The first column of A is: 
 [[-2]
 [ 1]]
Part 10:
 New B matrix is: 
 [[-1  1]
 [ 1  0]]
Part 11:
 The element-wise product between the first and second columns of A is: 
 [[6]
 [0]]
\end{verbatim}
\end{flushleft}
\end{document}
